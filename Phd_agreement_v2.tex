\documentclass[]{scrreprt}
\usepackage{graphicx}
\usepackage{subcaption}
\usepackage{epstopdf}
\usepackage{amsmath}
\usepackage{amsfonts}
\usepackage{amssymb}
\usepackage{tikz}
\usepackage{hyperref}
\usepackage{color}
\usepackage[toc,page]{appendix}
\usepackage{wrapfig}
\usepackage{verbatim}
%\usepackage{mathtools}
\usepackage{epigraph}
\usepackage{rotating}
\usepackage{cite}
\usepackage{wasysym}
\usepackage{adjustbox}

\usepackage{rotating}
%
%\usepackage[backend=bibtex]{biblatex}
%\addbibresource{Averaging.bib}
% \epigraphsize{\small}% Default
%\setlength\epigraphwidth{8cm}
%\setlength\epigraphrule{0pt}

\usepackage{etoolbox}
\usepackage{booktabs}
%%Table

%%
\makeatletter
\patchcmd{\epigraph}{\@epitext{#1}}{\itshape\@epitext{#1}}{}{}
\makeatother

\newtheorem{theorem}{Theorem}[section]
\newtheorem{lemma}[theorem]{Lemma}
\newtheorem{proposition}[theorem]{Proposition}
\newtheorem{corollary}[theorem]{Corollary}

\newenvironment{proof}[1][Proof]{\begin{trivlist}
		\item[\hskip \labelsep {\bfseries #1}]}{\end{trivlist}}
\newenvironment{definition}[1][Definition]{\begin{trivlist}
		\item[\hskip \labelsep {\bfseries #1}]}{\end{trivlist}}
\newenvironment{example}[1][Example]{\begin{trivlist}
		\item[\hskip \labelsep {\bfseries #1}]}{\end{trivlist}}
\newenvironment{remark}[1][Remark]{\begin{trivlist}
		\item[\hskip \labelsep {\bfseries #1}]}{\end{trivlist}}

\newcommand{\qed}{\nobreak \ifvmode \relax \else
	\ifdim\lastskip<1.5em \hskip-\lastskip
	\hskip1.5em plus0em minus0.5em \fi \nobreak
	\vrule height0.75em width0.5em depth0.25em\fi}

% Title Page
%\title{}
%\author{}
\newcommand*{\titleGM}{\begingroup % Create the command for including the title page in the document
	\hbox{ % Horizontal box
		\hspace*{0.2\textwidth} % Whitespace to the left of the title page
		\rule{1pt}{\textheight} % Vertical line
		\hspace*{0.05\textwidth} % Whitespace between the vertical line and title page text
		\parbox[b]{0.75\textwidth}{ % Paragraph box which restricts text to less than the width of the page
			
			{\noindent\Huge\bfseries PhD Agreement \\}\\[2\baselineskip] % Title
			{\large \textit{ Modular DC-DC Converter for future LVDC Distribution}}\\[4\baselineskip] % Tagline or further description
		%	\vspace{0.2\textheight}
			{\large Author: \textsc{Pavel Purgat }} \\[4\baselineskip]
		
		%	{\large \emph{Supervisor:} Dr. \textsc{ Jelena Popović}} \\
		%	{\large \emph{Promotor:} prof. \textsc{Pavol Bauer}} \\
		{ \large	\begin{tabular}{lll}
			
				Supervisor: & \ dr.\ ir.\ \textsc{J.}\ \textsc{Popovic} \\
				Promotor:	& prof.\ dr.\ ir.\ \textsc{P.}\ \textsc{Bauer}
				
			\end{tabular} }
			
			\vspace{0.5\textheight} % Whitespace between the title block and the publisher
			{\noindent  }% here should be text inside
				\includegraphics[width=0.3\linewidth]{../Pictures/TUDlogo}%\\[\baselineskip]

 % Publisher and logo 
		
			
		}}
			
		\endgroup}
	
	% Sets margins and page size
	\pagestyle{empty} % Removes page numbers
	\makeatletter % Need for anything that contains an @ command 
	\renewcommand{\maketitle} % Redefine maketitle to conserve space
	{ \begingroup \vskip 10pt \begin{center} \Huge {\bf \@title}
			\vskip 10pt \large \@author \hskip 20pt \@date \end{center}
		\vskip 10pt \endgroup \setcounter{footnote}{0} }
	\makeatother % End of region containing @ commands

\begin{document}
\titleGM

\begin{abstract}
	
\end{abstract}
        



\chapter{Introduction}

In past decade large penetration of power electronics into everyday life can be observed. This vast penetration lead to shift in characteristic of power distribution system from inductive and resistive towards capacitive and non-linear\cite{Boroyevich2007}. As a consequence of this vast implementation of the power electronic converters,  the field has reached a point in its development when it is marked as a mature engineering and research field\cite{Technology2015}. Resulting from this development power electronic is pushing forward new applications. A good example is reincarnation of the AC vs. DC debate\cite{IEEE2016}. 

%
It seems that in common consciousness the DC has lost a battle more than a century ago and passed out of all knowledge. However, the truth seems to be on the contrary, the DC has survived and dominated over AC in very specific applications such as telecommunication. From telecommunication it made its way into consumer electronics, and from consumer electronics to Microgrid research and application.  

%
With increasing penetration of renewable energy sources new challenges arise for the power distribution systems. The power systems distribution were designed for a situation with strictly defined power flow, and centralized power production. The renewable energy sources are diffuse in nature, which makes them distributed over large geographical areas. Furthermore, the renewable energy sources are also intermittent, meaning that the energy is dispersed over time periods. It has been discussed that high penetration of distributed energy sources can lead to overloading of the system. Therefore, in order to utilize more of the renewable energy sources the power distribution system needs an upgrade to become more flexible in its topology to allow for multi-directional flow of power both in space and time. %% The negative prices of electricity are a sign of badly designed system and market, and not a sign of more sustainable society.

Secondly, there is a strong trend in the world towards higher urbanization. This leads to concentration of large amounts of people in space restricted areas. Whereas, this has clear implications for social sciences, there are also implications for power distribution system. The concentration of humans in a small area leads to large concentration of energy utilization. This conditions higher power carrying capabilities of the system. High power carrying capability with high availability in AC system means serious oversizing of the network. This is in direct contradiction with  society interest, more efficient and sustainable energy usage. 

Society is asking for more efficient utilization of energies,  and part of the answer might be to introduce a low voltage distribution network based on DC. The envisioned distribution network should not suffer from the same shortcomings as the AC based distribution. And with the newly developed DC distribution network it should be easier to incorporate all different aspects of smart grids. 

Furthermore, developing a complex solution for a distribution in developed world can have impact on the electrification of rural areas in the developing world. Simplified and less general solutions can be adapted in order to provide electricity for countries where population has lower buying strength.

\section{Power Electronics in Low Voltage DC Distribution}

The power electronics is often regarded as an enabling technology in many applications. However, if the technology is termed as enabling, the application is usually regarded as a pulling force of the research. However, in case of low voltage DC distribution it seems that power electronic matured and is pushing a new application. Thus it may seem that there are both pulling and pushing forces in the research of power electronics in low voltage DC distribution.

Even though the power electronic seems to stand behind creation of this field, there seems to be strong pulling force for improvement of the current state-of-the-art power converters. The origin of this pull force can be identified in the main trends driving introduction of more power electronic devices in the distribution grid eg. renewable resources, urbanization, better energy utilization etc. These trends projects themselves as requirements for highly flexible power distribution system with multiple applications for power converters. In order to accommodate the prospective changes over considerable time span the solution needs to be not just flexible but also scalable and standardized. These transferable features of the future distribution system provide a sufficient pull for research in power electronic field.

When examining envisioned low voltage DC distribution network in particular, we can categorize the power converters into categories according to their application:
\begin{itemize}
	\item Renewable Energy Source integration converters
	\item Load converters
	\item Energy Storage converters
	\item Distribution converters
\end{itemize}

For this project the "distribution" converter is of interest. Behind the unspecific name is a DC-DC converter which can be deployed at different parts of the network and in general should have at least these functions:
\begin{itemize}
	\item Enables DC voltage stepping
	\item DC power and/or DC voltage regulation
	\item Dynamics decoupling of interfaced systems (eg. connection between bipolar and single-bus architectures)
	\item Bidirectional DC fault isolation
\end{itemize}

The description of the future  low voltage DC distribution network and the role of the power electronics in it, can be transformed into following converter requirements:

\begin{itemize}
	\item Bidirectional power flow
	\item Smart metering and communication functions \cite{Liserre2016}
	\item High partial load efficiency and small stand-by losses
	\item Long Useful life
	\item N+1 redundancy
	\item Easy maintenance 
	\item Minimal Costs for life time usage (? how to put nicely that it is not just about capital investment but also about maintenance costs ?)
	\item Environment Friendly (Life-cycle analysis as a measure?\footnote{Such as: \href{https://www.pre-sustainability.com/electric-vehicles-are-best-for-green-mobility-myth-or-not}{www.pre-sustainability.com/coming-soon-sustainability-mythbusters}})
\end{itemize}




The most natural way for solving a problem is to decompose it into smaller sub-parts, solve each sub-part individually and afterwards connect the pieces. %This approach is also reflected in Adam Smith's "laissez faire" concept. Where via labour division much higher productivity is achieved. 
This approach is adopted in many research fields and can have different names. In power electronic this approach is referred to as modular topology,design or manufacturing. While modularity is widely studied in frame of product architecture, in power electronic field this term is used rather vaguely. Which can lead to considerable confusion.

The modular product architecture can be very beneficial if we either want to produce large variety of product at low-cost and/or there is need for independent design of a module from its function. Modular approach allows the producer to not only change product at one specific point in time, but also over a large time span. The product architecture  has far reaching implications  for maintenance and upgrades of the product. The modular product architecture encourages economy of scale based on standardization, increase in flexibility, outsourcing and many other benefits.


The modularity is not a new concept in power electronics. On the contrary there seems to be a maze of the research work, which makes it hard to rigorously study the impact of modularity on the improvement of power converters. Furthermore, the modular approach was never widely adapted by the industry which leads to lack of real-world data and experience. (?)


\section{Conclusion}

In the above text a new approach towards power distribution was briefly described. The role of power electronic devices was emphasized through out the text. As was described the modular concept can potentially hold many advantages. In spite of that, this concept was not as of now picked up by the industry in any significant way. Therefore following conclusions are drawn, which are implications for the aim and objectives of this project:
\begin{itemize}
	\item low voltage DC distribution is a plausible solution for the problems arising with increased integration of renewable energy sources
	\item low voltage DC distribution is used as a system of choice for very specific problems, which implies that the converters should be versatile in functionality and power level in order to create price competitive solution
	\item modular architecture and production of power converters can provide necessary framework for the successful adaptation of the low voltage DC  distribution
	\item modularity as a term describing cascaded structures of converters is widely used in available literature, however there seems to be space for extra gains beyond the cascaded electrical structure
\end{itemize}

\chapter{Aim and Research Objectives}

\section{Aim} Develop a concept modular DC-DC converter for low voltage DC distribution network.
 
 \vspace{10mm}

\section{Research Questions}

\begin{enumerate}
	\item How the state-of-the-art modular converters can be classified ? And how can modularity of the converters be defined such that will respect the specifics of power electronics and the supply chain in the low voltage distribution market ? 
	\subitem Based on the definition and classification of modular converters, what are the challenges on the component level for the application of low voltage DC distribution ?
	\item Can the modularity be quantified, such that this measure of modularity can help to decide where modular approach is beneficial and where is a breaking point where more integral product is more beneficial ?
	\item How can a design process be developed, such that a family of DC-DC network converters can be easily designed and manufactured while respecting the requirements of the market and the supply chain of the producer ? 
	\item What is the $\text{CO}_2$ footprint of the proposed family of DC-DC network converters and how it can be optimized by utilization of more environment friendly materials and manufacturing processes ? 
\end{enumerate}


\section{Research Objectives}

\begin{enumerate}
	\item Analyse present concepts of modular converters used in power electronic converter design and manufacturing. The concepts of modularity are to be analysed and categorized. Based on the analysis a classification of the modular converters is to be established. The classification will be used to create a definition of modular converter for low voltage DC distribution. This definition will encompass both design and manufacturing process. % Furthermore, the concepts of modularity are to be enriched by modularity in manufacturing. Creating a basis for truly modular converter beyond the cascaded converter. 
	\item  Based on the definition of modularity a design process which exploits the benefits of modularity is to be developed. This design process takes into account both specifics of the market(electric power distribution for secondary customers) and supply chain of the power electronic converter manufacturers and power electronic enabled solutions providers. 
	
	The design process should allow for design of concept family of modular DC-DC converters, such that they are easy to optimize for cost while maintaining wide variety of products and ability for fast product change.
	\item Environmentally friendly materials and manufacturing processes are preferred. Based on the life-cycle analysis the conceptual family of the converters will be evaluated in terms of $\text{CO}_2$ footprint. This step represents a direct effort towards decarbonizing the industry. 
\end{enumerate}

\section{Expected Contributions}

\begin{enumerate}
	\item Classification of modular converters and definition of modularity for converters in low voltage power distribution.
	\item Identification of challenges and limits of power electronic converters in low voltage power distribution.
	\item Measure of modularity as a design tool.
	\item Life-cycle analysis of the modular DC-DC converters, identifying the weakest points of the converters manufacturing process and materials used in terms of $\text{CO}_2$
\end{enumerate}
\newpage
\section{Assumptions and Constraints}

Assumptions on the system level:
\begin{itemize}
	\item LVDC network connecting multiple houses, with unspecific architecture 
	\item nominal voltage level is assumed to be 350 V
	\item other voltage levels include 48 V, 700 V, 1400 V
	\item network is at least as efficient as AC (?)
	\item large penetration of res
\end{itemize}

Constraints:
\begin{itemize}
	\item limited input from industry fellows (?)
	\item application is restricted to LV DC distribution network
\end{itemize}






\chapter{Strategy and planning}






\section{Doctoral Education}

\begin{table}[h!]
	\centering 
	\begin{adjustbox}{width=1\textwidth}
		\small
		\begin{tabular}{c c c} 
			\multicolumn{3}{c}{Doctoral Education Areas} \\ 
			\cmidrule(l){1-3}
		%	\multicolumn{2}{c}{Half-bridge } & \multicolumn{2}{c}{Full-bridge } \\
			Research competences and  & Discipline related skills & Transferable competences and    \\ % Column
			skills & (15 GSC) & skills \\
			(15 GSC) &        & (15 GSC) \\
			\midrule % In-table horizontal line
			\midrule
		 Performing and learning  & Summer School(5 GCS) & Attending GS courses like: \\
		 through on the job activities like: & PhD research School (3 GSC) & (C9.M1)PhD Startup \\
		 - Presenting and interacting & ECPE Tutorials & (C11) information, computing \\
		 with scientific community & and Workshops & and language\\
		 - Supervision of MSc students & Working with partner &  (C10) Professional development  \\
		 -Writing and publishing & organizations  &  (C12)  Effective presentation\\
		  research findings &  & (C13) Cooperation/teamwork \\
			\bottomrule % Bottom horizontal line
		\end{tabular}
	\end{adjustbox}
	%	\caption{Table caption text} % Table caption, can be commented out if no caption is required
	\label{tab:template} % A label for referencing this table elsewhere, references are used in text as \ref{label}
\end{table}
%Transferable skills: C11.M14, C13.M6,C8.M4,C12.M4,C12.M9,C12.M11,C10.M3,C10.M4,C10.M7-last three are for last year

\section{Supervision}
It has been agreed that a weekly meeting will take place between the PhD student and the daily supervisor (Dr. Jelena Popovic). Additionally, a monthly meeting will take place with promotor prof.Pavol Bauer.

\appendix
\section*{Timelines}
\begin{sidewaysfigure}[h!]
	\centering
	\includegraphics[height=.9\textheight,width=.9\textwidth]{Pictures/PhDbigplan}
	\caption{Timeline of the PhD project.}
	\label{fig:gantt}
\end{sidewaysfigure}
\begin{sidewaysfigure}[h!]
	\centering
	\includegraphics[height=.9\textheight,width=.9\textwidth]{Pictures/Phd06months}
	\caption{Timeline of the PhD project.}
	\label{fig:Phd06months}
\end{sidewaysfigure}
	\newpage
	%% Bibliography
	\bibliography{MyCollection}{}
	\bibliographystyle{plain}
	
\end{document}  